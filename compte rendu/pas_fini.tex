\documentclass{article}

\usepackage{amssymb}
\usepackage{amsmath}
\usepackage[french]{babel}
\usepackage[utf8]{inputenc} 
\usepackage[T1, T2A]{fontenc}

\usepackage{listings}

\usepackage[a4paper,left=2cm,right=2cm,top=2cm,bottom=2cm]{geometry}

\usepackage{setspace}

\usepackage{graphicx}

\usepackage{stmaryrd}
\usepackage{float}
\usepackage{colonequals}

\setlength{\parindent}{0cm}
\setlength{\parskip}{1ex plus 0.5ex minus 0.2ex}
\newcommand{\hsp}{\hspace{20pt}}
\newcommand{\HRule}{\rule{\linewidth}{0.5mm}}
\newcommand*{\logeq}{\ratio\Leftrightarrow}

\title{Compte rendu}
\author{Arthur Lacoin - Timothé Rios}

\date{}
    
\begin{document}

\begin{titlepage}
    \begin{sffamily}
    \begin{center}

        \textsc{\LARGE Lycée Lakanal}~\\[2cm]

        \HRule \\[0.4cm]
        { \huge \bfseries Simulation d'épidémies à l'aide d'automates cellulaires\\[0.4cm] }
        \HRule \\[2cm]
        \textsc{\Large Arthur Lacoin - Timothé Rios}\\[2cm]

    \end{center}
\end{sffamily}
\end{titlepage}

\tableofcontents

\newpage
\paragraph{Abstract\\}
	Within the theme of transport, we have looked at how an epidemic spreads. To be able to represent this phenomenon, we chose to use a cellular automaton. A cellular automaton is a matrix of cells whose states vary over time. Here, at every moment, each cell can be Healthy, Sick, Deceased or Healed, according to its state as well as that of its neighbours at the moment before. Thus, with the information of an initial state, it is possible to materialize the spread of an epidemic. Of course, in order to accurately represent reality, it is necessary to make this model more complex by taking into account many factors that play a role in how a disease spreads.\\[2cm]
	
\begin{center}
\LARGE
\bf
\textsc{Introduction}~\\[1cm]
\end{center}

	L'étude de la propagation des maladies est une science qui remonte à la Grèce Antique, lorsque Hippocrate, au quatrième siècle avant Jésus-Christ, créa les termes d'endémie et d'épidémie afin de qualifier respectivement des maladies liées à des endroits et à des périodes donnés. Mais cette science ne se développa vraiment qu'à partir de 1854, quand John Snow, un médecin britannique, étudia la propagation de l'épidémie de choléra dans les quartiers de Londres. Ce sont là les prémices de l'épidémiologie, l'étude du transport des infections. De nos jours, les méthodes ont évolué et la méthode différentielle, que l'on présentera plus loin, est la plus fréquemment utilisée. Nous nous sommes aussi intéressés ici à la simulation par automates cellulaires.

\section{Première version}

\subsection{Présentation}
\paragraph{Définition formelle\\[0.2cm]}
	Un automate cellulaire est une matrice de cellules, chacune ayant un état appartenant à un ensemble prédéfini. L'état de chaque cellule peut varier au cours du temps suivant une fonction de transfert : l'état de la matrice à l'instant $t+1$ dépend ainsi de son état à l'instant $t$. L'automate doit donc posséder un état initial, c'est-à-dire la matrice des états initiaux des cellules. Même si son principe de base est simple, l'automate cellulaire se complexifie grandement lorsque la taille de la grille augmente, ce qui en fait un modèle couramment utlisé dans l'étude des systèmes complexes.

\paragraph{Historique\\[0.2cm]}
	Les automates cellulaires sont plutôt récents. Ils ont été mis au point par John Von Neumann dans son étude des systèmes auto-réplicatifs dans les années 1940. Cette notion a été grandement popularisée par le "jeu de la vie" de John Conway, un automate cellulaire en deux dimensions paru dans les années 1970. À ce jour, les automates cellulaires ont de nombreuses applications dans divers domaines :
	\begin{itemize}
	\item diffusion d'un gaz en s'appuyant sur les équations de Navier-Stokes
	\item simulation des feux de forêts
	\item simulation du trafic routier
	\end{itemize}

\paragraph{Notre automate cellulaire\\[0.2cm]}
	Notre automate cellulaire est un automate en deux dimensions qui vise à simuler la propagation d'une épidémie. Les cellules ont donc un état parmi les quatre suivants : Sain, Malade, Guéri, Mort. L'état Sain est l'état initial que chaque cellule sauf une possède à la première étape de la simulation. Lorsqu'une cellule est saine, elle peut tomber malade avec une probabilité $p_1$ s'il y a des malades dans son entourage. L'état Malade est au départ donné à une seule cellule, le "patient zéro". Une cellule malade peut soit mourir avec une probabilité $p_2$ soit guérir avec une probabilité $p_3$. Les cellules mortes ou guéries restent dans cet état indéfiniment, l'hypothèse prise étant qu'une cellule guérie est immunisée et ne peut plus attraper la maladie. Enfin, la simulation s'arrête lorsqu'il n'y a plus de cellule malade, car l'état de l'automate ne peut alors plus varier.
	
\paragraph{Calcul des probabilités\\[0.2cm]}
	Les probabilités d'infection, de soin et de mort que nous avons utilisé ont été calculées à partir de paramètres empiriques tirés d'études réelles. On utilisera en particulier $\tau$, le temps moyen qu'un individu reste infecté, et $R_0$, le nombre moyen d'individus qu'un malade contamine durant son temps de contamination.
	
	On cherche d'abord à déterminer la probabilité d'infection. On se place donc dans $(\Omega, P)$ un espace probabilisé fini. On étudie un individu malade entouré de $n$ voisins sains. On considère que l'infection d'un individu sain est indépendante des autres. \\
	On établira d'abord un résultat préliminaire.\\
	Soit $X, Y$ deux variables aléatoires réelles sur $\Omega$ et à valeur dans $\llbracket 0, n \rrbracket, n \in \mathbb{N}$, telles que $\forall k \in \llbracket 0, n \rrbracket, X$ sachant que $Y = k$ suit une loi binomiale de paramètres $(n-k, p), p \in \left]0, 1\right[$.\\ \underline{Alors $E(X) = (n - E(Y))p$.}\\[0.2cm]
	On a donc $\forall k \in \llbracket 0, n \rrbracket, E_{\{Y = k\}}(X) = (n - k)p$. Par formule de l'espérance totale, 
	\begin{align*}
	E(X) &= \sum_{k=0}^n P(Y = k)E_{\{Y = k\}}(X)  \\
	     &= \sum_{k=0}^n P(Y=k)(n-k)p \\
		 &= np\sum_{k=0}^n P(Y = k) - p\sum_{k=0}^n kP(Y=k) \\
		 &= np - pE(Y) \\
		 &= p(n - E(Y))
	\end{align*}
	Soit $k \in \llbracket 1, \tau \rrbracket, X_k$ est la variable aléatoire donnant le nombre d'individus qui ont été infectés au jour $k$.\\
	$X_1 \hookrightarrow B(n, p)$ car on infecte les individus indépendamment.\\ 
	Au jour $k$ il reste $$n - \sum_{i = 0}^{k-1} X_i$$ individus sains.\\[0.2cm] \underline{Montrons par récurrence que $\forall k \in \llbracket 1, \tau \rrbracket, E(X_k) = n(1-p)^{k-1}p$}\\[0.2cm]
	
	\underline{$\forall k \in \llbracket 1, \tau \rrbracket$, notons $P(k) = "\forall i \in \llbracket 1, k \rrbracket, E(X_i) = n(1-p)^{i-1}p"$}.\\[0.2cm]
	\underline{$P(1)$ est vraie} car $X_1 \hookrightarrow B(n, p)$.\\[0.2cm]
	\underline{Ou bien $\tau = 1$} et c'est fini.\\[0.2cm]
	\underline{Ou bien $\tau \ge 2$} et on \underline{suppose que $P(k)$ est vraie pour $k \in \llbracket 1, \tau - 1 \rrbracket$. Montrons que $P(k+1)$ est vraie.}\\[0.2cm]
	$\forall i \in \llbracket 1, k \rrbracket, E(X_i) = n(1-p)^{i-1}p$ par $P(k)$. Il reste à calculer $E(X_{k+1})$.\\
	\begin{align*}
		E(X_{k+1}) &= p(n - \sum_{i = 1}^k E(X_i)) \\
		           &= p(n - \sum_{i = 1}^k n(1-p)^{i - 1}p) \\
		           &= np(1-p)^k
	\end{align*}
	On conclut par récurrence multiple finie.\\
	Armés de ces résultats, nous pouvons quasiment calculer $p$, car on a le nombre moyen d'infectés par un malade au cours de sa période d'infection, $R_0$. On a donc : \\ 
	\begin{align*}
		R_0 &= \sum_{k=1}^\tau E(X_k) \\
		    &= \sum_{k=1}^\tau np(1-p)^{k-1}\\
		    &= np\frac{1 - (1-p)^\tau}{p}\\
		    &= n(1 - (1-p)^\tau)
	\end{align*}
	
	Ce qui nous amène par de brefs calculs à la conclusion : \\
	$$p = 1 - \sqrt[\tau]{\frac{n - R_0}{n}}$$.\\
	
	On cherchera désormais à calculer la probabilité de ne plus être malade, c'est à dire ou bien de guérir ou bien de mourir, c'est à dire $p_{\text{mort}} + p_{\text{guérison}} = q$. Si l'on considère un individu qui est infecté au jour $0$, et que l'on considère la variable aléatoire $X$ donnant le jour de guérison(sachant qu'on ne peut guérir le jour de son infection), $\forall k \in \mathbb{N}^*, P(X=k) = (1-q)^{k-1}q$.\\
	X suit donc une loi géométrique et son espérance, c'est à dire le nombre de jour où l'individu restera malade($\tau$) vaut par propriété $\frac{1}{q}$.\\
	On a donc $$ q = \frac{1}{\tau}$$\\
	

Enfin pour calculer la probabilité de mourir, il suffit de faire $mq$, où $m$ est la mortalité de la maladie, et la probabilité de guérir vaut $(1-m)q$. \\
Les résultats réels sont récapitulés dans le tableau suivant :

\begin{tabular} {| p{2cm} | p{2cm}| p{2cm} | p{2cm} |}
	\hline
	Maladie & $p_{infection}$ & $p_{mort}$ & $p_{guerison}$ \\ \hline
	Ebola & $0.0187$ & $0.0818$ & $0.00909$ \\ \hline
	Grippe & $0.0406$ & $0.0002$ & $0.1998$ \\
	\hline
\end{tabular}

\subsection{Résultats}

Voici quelques étapes d'une simulation de notre automate. Le foyer originel -la première cellule infectée- est placée au centre de la grille à l'étape 1. Sur ces images,représentant la grille de l'automate, les cellules saines sont représentées en blanc, celles malades sont rouges, celle guéries sont vertes et celles mortes sont noires.

\end{document}